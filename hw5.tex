> ##Homework #5
> Vu Dinh

---

###Problem 1. True or False? Justify.

a. If $s_n \rightarrow c$ and $t_n \rightarrow c$ , then $s_n = t_n \quad\forall n \in \mathbb N$. 

**False**. Counterexample:  
> Consider $(s_n) = \frac{1}{n^2}$ and $(t_n) = \frac{1}{n^2+4}$.

> * $\lim(s_n)=0$ and $\lim(t_n) = 0$, yet

> * For any $n \in \mathbb N$, $s_n \ne t_n$.

------
b. If $\forall \epsilon > 0, \exists N \in \mathbb N$ such that $n \geq N \text{ implies } s_n < \epsilon \text{, then } s_n \rightarrow 0$.

**False**. The qualification does not require $s_n$ to be a positive sequence, so this statement is trivially true if $s_n$ is eventually negative. If this is, however, a typo, then the statement is correct per the definition of *limit of sequence*.

---
c. If $s_n \rightarrow s$ and $s_n \rightarrow t$, then $s = t$.

**True**. We proved a theorem in class to justify the uniqueness of limits of sequences.

---
d. Every convergent sequence is bounded.

**True**, although my justification may be sketchy.
> Lemma: Every sequence is either **monotonic** or **periodic**.
> If a sequence is periodic, then there are a finite number of values that the sequence terms can hold, regardless of the value of $n$. This means that the set $S$ that contains all the values of the sequence $s_n$ is finite.

